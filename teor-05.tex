\subsection{Энергия колебаний ограниченной струны. Теорема единственности для смешанной краевой задачи для уравнения колебаний струны.}

\paragraph{Энергия колебаний ограниченной струны.}

Найдем выражение для энергии поперечных колебаний струны $E = K + U$, где $K$ --- кинетическая и $U$ - потенциальная энергия. Элемент струны $dx$, движущийся со скоростью $v = u_t$, обладает кинетической энергией 
\begin{equation*}
	\frac{1}{2} m v^2 = \frac{1}{2} \rho(x) \, dx(u_t)^2 \quad (m = \rho \, dx)
\end{equation*}

Кинетическая энергия всей струны равна 
\begin{equation}
	K = \frac{1}{2} \int \limits_{0}^{l} \rho(x) [u_t(x, t)]^2 \, dx.
\end{equation}

Потенциальная энергия поперечных колебаний струны, имеющей при $t = t_0$ форму $u(x, t_0) = u_0(x)$, равна работе, которую надо совершить, чтобы струна перешла из положения равновесия в положение $u_0(x)$. Пусть функция $u(x, t)$ дает профиль струны в момент $t$, причем
\begin{equation*}
	u(x, 0) = 0, \quad u(x, t_0) = u_0(x).
\end{equation*}
Элемент $dx$ под действием равнодействующей сил натяжения 
\begin{equation*}
	T \frac{\partial u}{\partial x} \Big|_{x + dx} - T \frac{\partial u}{\partial x} \Big|_{x} = T u_{xx} \, dx
\end{equation*}
за время $dt$ проходит путь $u_t(x, t) \, dt$. Работа, производимая всей струной за время $dt$, равна
\begin{equation*}
	\Bigg\{\int \limits_{0}^{l} T_0 u_{xx} u_t \, dx \Bigg\} \, dt = \Bigg\{T_0 u_x u_t \Big|_{0}^{l} - \int \limits_{0}^{l} T_0 u_x u_{xt} \, dx \Bigg\} \, dt =  \Bigg\{-\frac{1}{2} \frac{d}{dt} \int \limits_{0}^{l} T_0 (u_x)^2 \, dx + T_0 u_x u_t \Big|_{0}^{l}\Bigg\} \, dt.
\end{equation*}
Интегрируя по $t$ от $0$ до $t_0$, получаем
\begin{equation*}
	-\frac{1}{2} \int \limits_{0}^{l} T_0(u_x)^2 \, dx \Big|_{0}^{t_0} + \int \limits_{0}^{t_0} T_)0 u_x u_t \Big|_{0}^{l} \, dt = -\frac{1}{2} \int \limits_{0}^{l} T_0 [u_x(x, t_0)]^2 \, dx + \int \limits_{0}^{t_0} T_0 u_x u_t \Big|_{0}^{l} \, dt.
\end{equation*}

Нетрудно выяснить смысл последнего слагаемого правой части этого равенства. Действительно, $T_0 u_x|_{x = 0}$ есть величина натяжения на конце струны $x = 0; \, u_t(0, t) \, dt$  --- перемещение этого конца, а интеграл 
\begin{equation}
	\int \limits_{0}^{t_0} T_0 u_x u_t|_{x = 0} \, dt
\end{equation}
представляет работу, которую надо затратить на перемещение конца $x = 0$. Аналогичный смысл имеет слагаемое, соответствующее $x = l$. 

Если концы струны закрепленны, то работа на них будет равна нулю (при этом $u(0, t) = 0, u_t(0, t) = 0$). Следовательно, при перемещении закрепленной на концах струны из положения равновесия $u = 0$ в положение $u_0(x)$ работа не зависит от способа перехода струны в это положение и равна 
\begin{equation}
	-\frac{1}{2} \int \limits_{0}^{l} T_0[u'_0(x)]^2 \, dx, 
\end{equation}
потенциальной энергии струны в момент $t = t_0$ с обратным знаком.

Таким образом, полная энергия струны равна
\begin{equation}
	E = \frac{1}{2} \int \limits_{0}^{l} [T_0(u_x)^2 + \rho(x)(u_t)^2] \, dx.
\end{equation}

Совершенно аналогично может быть получено выражение для потенциальной энергии продольных колебаний стержня. Впрочем, его можно получить также, исходя из формулы для потенциальной энергии упругого стержня
\begin{equation*}
	U = \frac{1}{2} k(\frac{l - l_0}{l_0})^2 l_0,
\end{equation*}
где $l_0$ --- начальная длина стержня, $l$ --- конечная длина. Отсюда непосредственно следует 
\begin{equation*}
	U = \frac{1}{2} \int \limits_{0}^{l} k(u_x)^2 \, dx.
\end{equation*}

\paragraph{Теорема единственности для смешанной краевой задачи для уравнения колебаний струны.}
