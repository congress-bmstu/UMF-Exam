\subsection{Функции, гармонические в области. Теорема о среднем значении для гармонических
функций. Принцип максимума.}

\paragraph{Гармонические в области функции}
ОПРЕДЕЛЕНИЕ. Вещественнозначная функция $u(x)$ класса $\mathcal{C}^2 (G)$ называется 
\emph{гармонической в области $G$}, если она удовлетворяет уравнению Лапласа $\Delta u = 0$
в этой области.

При $n=1$ гармонические фунцкции своядтся к линейным функциям, и потом их теория интереса
не представляет. Поэтому в дальнейшем будем считать $n \geqslant 2$. Нетривиальным примером
гармонической функции при $\| \vec{x} \| \neq 0$ является фундаментальное решение оператора
Лапласа:
\begin{align*}
  \varepsilon_2 (x) &= \dfrac{1}{2\pi} \ln \| \vec{x} \|, n = 2, \\
  \varepsilon_n (x) &= - \dfrac{1}{(n-2) \sigma_n} \| \vec{x} \|^{-n+2}, n \geqslant 3,
\end{align*}
этот факт легко доказать, если перейти к обобщённым сферическим координатам (ну а для малых
размерностей это будет сооствественно полярные и обычные сферические). В этих координатах
$\| \vec{x} \| = r$, а все члены оператора Лапласа, зависящие от производных по углам, будут
заведомо равны нулю. Тогда оставшийся член будет равен:
\[
  \Delta \varepsilon_n = 
  \dfrac{\partial^2 \varepsilon_n}{\partial r^2} 
  + \dfrac{n-1}{r} \dfrac{\partial \varepsilon_n}{\partial r}, 
\]
в области $\|\vec{x}\| \neq 0$, очевидно, это равно нулю.
