\subsection{Единственность решения краевой задачи (внутренней и внешней) для уравнения Лапласа.}
%autor: Сеня

Рассмотрим уравнение Лапласа $\Delta u = 0$.

Функция $u$ называется гармонической в области $T$, если она непрерывна в этой области вместе со своими производными до 2-го порядка и удовлетворяет уравнению Лапласа. 

\begin{theorem}
	Первая (Дирихле) внутренняя краевая задача для уравнения Лапласа не может иметь двух различных решений\footnote{Т.-С. стр. 317}.
\end{theorem}


\begin{proof}
	Допустим, что существуют решения\footnote{Непрерывные в замкнутой области} $u_1$ и $u_2$. Рассмотрим разность $u = u_1 - u_2$.
	
	Убедимся в том, что $u \equiv 0$. Если функция $u \not \equiv 0$ и хотя бы в одной точке $u > 0$, то она должна достигать положительного максимального значения внутри области, что невозможно\footnote{см. раздел \ref{max}}. Совершенно так же доказывается, что функция $u$ не может принимать внутри области отрицательных значений. Отсюда следует, что 
	\begin{equation*}
		u \equiv 0.
	\end{equation*}
\end{proof}

%TODO: оформить все по-другому

\textit{Первая (Дирихле) внешняя краевая задача для уравнения Лапласа не может иметь двух различных решений}.\footnote{Владимиров В.С. стр. 295}

Согласно следствию из принципа максимума, если функция $u$ гармоническая вне области $G$, и $u(\infty) = 0$, то так же можно говорить, что 
\begin{equation*}
	\abs{u(x)} \leqslant \max\limits_{\xi \in S}{\abs{u(\xi)}}, \quad x \in \bar{G_1}, G_1 = \mathbb{R}^n \setminus \bar{G}  
\end{equation*}

Поэтому единственность доказывается аналогично внутреннему случаю. 

\textit{Докажем, что решение второй (Нейман) внутренней краевой задачи определяется с точностью до произвольной постоянной.}\footnote{Т.-С. стр. 325}

Рассмотрим функцию $u = u_1 - u_2$, где $u_1, u_2$ - решения. Тогда на границе $\Sigma$ выполняется $\frac{\partial u}{\partial n} \Big|_{\Sigma} = 0$.

Полагая в первой формуле Грина (см. \eqref{first_green_formula} в следующем пункте) $v = u$ и учитывая соотношения $\Delta u = 0$ и $\partial u / \partial n |_{\Sigma} = 0$, получаем 
\begin{equation*}
	 \iiint \limits_{T} \Bigg[\Bigg(\frac{\partial u}{\partial x}\Bigg)^2 + \Bigg(\frac{\partial u}{\partial y}\Bigg)^2 + \Bigg(\frac{\partial u}{\partial z}\Bigg)^2\Bigg] \, d\tau = 0.
\end{equation*}

Отсюда в силу непрерывности функции $u$ и ее первых производных следует 
\begin{equation*}
	\frac{\partial u}{\partial x} = \frac{\partial u}{\partial y} = \frac{\partial u}{\partial z} \equiv 0, \quad \text{ т.е. } u \equiv const,
\end{equation*}
что и требовалось доказать. 

Изложенный метод применим и в случае неограниченной области для функций, удовлетворяющих требованиям регулярности на бесконечности.

%TODO: Объяснить что такое регулярность 