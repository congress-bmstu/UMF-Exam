\subsection{Единственность решения краевой задачи (внутренней и внешней) для уравнения Лапласа.}

При исследовании стационарных процессов различной физической природы (колебания, теплопроводность, диффузия и др.) обычно приходят к уравнениям эллиптического типа. Наиболее распространенным уравнением этого типа является уравнение Лапласа
\begin{equation*}
	\Delta u = 0. 
\end{equation*}

Функция $u$ называется гармонической в области $T$, если она непрерывна в этой области вместе со своими производными до 2-го порядка и удовлетворяет уравнению Лапласа. 

Рассмотрим некоторый объем $T$, ограниченный поверхностью $\Sigma$.
Краевые задачи формулируются следующим образом.

\textit{Найти функцию $u(x, y, z)$, удовлетворяющую внутри $T$ уравнению Лапласа и граничному условию, которое может быть взято в одном из следующих видов:} 
\begin{enumerate}
	\item $u = f_1 $ на $ \Sigma $ \quad \quad \quad \quad \quad \quad \quad \quad \texttt{(первая краевая задача)},
	
	\item $\frac{\partial u}{\partial n} = f_2$ на $ \Sigma $ \quad \quad \quad \quad \quad \quad \quad \texttt{(вторая краевая задача)},
	
	\item $\frac{\partial u}{\partial n} + h(u - f_3) = 0$ на $ \Sigma $ \quad \quad \texttt{(третья краевая задача)},
\end{enumerate}
\textit{где $f_1, f_2, f_3, h$ --- заданные функции, $\partial u / \partial n$ --- производная по внешней нормали к поверхности $\Sigma$}.

Первую краевую задачу для уравнения Лапласа часто называют задачей Дирихле, вторую --- задачей Неймана.

Если ищется решение в области $T_0$, внутренней (или внешней) по отношению к поверхности $\Sigma$, то соответствующую задачу называют внутренней (или внешней) краевой задачей. 

\textit{Первая внутренняя краевая задача для уравнения Лапласа не может иметь двух различных решений}.

Допустим, что существуют две различные функции $u_1$ и $u_2$, являющиеся решениями задачи, т.е. функции, непрерывные в замкнутой области $T + \Sigma$, удовлетворяющие внутри области уравнению Лапласа и на поверхности $\Sigma$ принимающие одни и те же значения $f$. Разность этих функций $u = u_1 - u_2$ обладает следующими свойствами:
\begin{enumerate}
	\item $\Delta u = 0$ внутри области $T$;
	
	\item $u$ непрерывна в замкнутой области $T + \Sigma$;
	
	\item $u|_{\Sigma} = 0.$
\end{enumerate}

Функцию $u(M)$, таким образом, непрерывна и гармонична в области $T$ и равна нулю на границе. Как известно, всякая непрерывная функция в замкнутой области достигает своего максимального значения (п. \ref{maximum_principle}). Убедимся в том, что $u \equiv 0$. Если функция $u \not \equiv 0$ и хотя бы в одной точке $u > 0$, то она должна достигать положительного максимального значения внутри области, что невозможно. Совершенно так же доказывается, что функция $u$ не может принимать внутри $T$ отрицательных значений. Отсюда следует, что 
\begin{equation*}
	u \equiv 0.
\end{equation*}

\textit{Докажем, что решение второй внутренней краевой задачи определяется с точностью до произвольной постоянной.}

Доказательство проведем при дополнительном предположении, что функция $u$ имеет 