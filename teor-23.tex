\subsection{Собственные значения и собственные функции задачи Штурма -- Лиувилля для цилиндра. Краевые задачи для уравнений Лапласа и Пуассона в ограниченном цилиндре}
\subsubsection{Штурм -- Лиувилль}
%aut: Slava
%ref: Свешников, стр. 121 (147 2ed.)
Рассматривается задача
\[
  \begin{cases}
    \Delta u + \lambda u = 0, & 0 < r < a, \ 0\leqslant \varphi \leqslant 2\pi,
    \ 0 < z < l,\\
    \alpha u_r(a, \varphi, z) + \beta u(a, \varphi, z) = 0,\\
    \alpha_1 u_z(r,\varphi, 0) - \beta_1 u(r,\varphi, 0) = 0,\\
    \alpha_2 u_z(r,\varphi, l) + \beta_2 u(r,\varphi, l) = 0.
  \end{cases}
\]
Напомним\footnote{См. раздел \ref{sec:14}.}, 
\begin{equation}
  \label{eq:laplace_cyl}
  \Delta u(r,\varphi,z) = \Delta_{r\varphi} u + u_{zz},
\end{equation}
где $ \Delta_{r \varphi} $ --- оператор Лапласа на плоскости $ (r, \varphi) $.

Разделим переменные --- $ u = v(r, \varphi) Z(z) $ --- и получим соотношение 
\[
  \frac{\Delta_{r\varphi} v + \lambda v}{v} = - \frac{Z''}{Z} =: \nu \geqslant 0.
\]
Так задача Штурма -- Лиувилля распалась на две задачи Штурма -- Лиувилля: 
\[
  \begin{cases}
    Z'' + \nu Z = 0, \quad 0 < z < l,\\
    \alpha_1Z'(0) -\beta_1 Z(0) = 0,\\
    \alpha_2 Z'(l) + \beta_2 Z(l) = 0.
  \end{cases}\quad
  \begin{cases}
    \Delta v + \varkappa v = 0, \quad 0 < r < a, \ 0 \leqslant \varphi \leqslant
    2\pi,\\
    \alpha v_r(a,\varphi) + \beta v(a,\varphi) = 0,
  \end{cases}
\] %FIXME: в учебнике вместо v_r стоит v_z
где $ \varkappa := \lambda - \nu $. Первая (одномерная) задача была решена в
разделе \ref{sec:S-L}, вторая (круговая) --- в разделе \ref{sec:20}\footnote{При
данных условиях $ \varkappa \geqslant 0 $. Иной случай
разобран ниже.}. 
%TODO: откуда знаем, что \varkappa > 0?

Собственные функции имеют вид 
\[
  u_{knm}(r, \varphi, z) = J_n \left( \sqrt{\varkappa_{kn}}r \right) (A_{kn}
\cos n\varphi + B_{kn} \sin n\varphi) Z_m(z),
\]
а собственные значения вычисляются по формуле $ \lambda_{knm} = \varkappa_{kn} +
\nu_m$.



\subsubsection{Лаплас}
%ref: Пикулин, стр. 32
\label{sec:laplace_cylinder}
Физические задачи почти во всех случаях обладают азимутальной симметрией, то есть ищется
функция $ u(r, z) $, поскольку граничные условия не
зависят от $ \varphi $. Это сильно упрощает задачу. Рассмотрим, однако, общий
случай\footnote{См. также формулу \eqref{eq:laplace_cyl}. Минус при $ \alpha_1 $
указывает на то, что внешняя нормаль направлена против оси $ z
$.}
\[
  \begin{cases}
    \Delta u = 0, & 0 < r < a, \ 0\leqslant \varphi \leqslant 2\pi,
    \ 0 < z < l,\\
    \alpha u_z(r, \varphi, l) + \beta u(r,\varphi, l) = f(r, \varphi),\\
    -\alpha_1 u_z(r,\varphi, 0) + \beta_1 u(r,\varphi, 0) = g(r, \varphi),\\
    \alpha_2 u_r(a,\varphi, z) + \beta_2 u(a,\varphi, z) = h(\varphi, z).
  \end{cases}
\]
Представим решение в виде\footnote{Разделение переменных здесь не поможет
хотя бы потому, что все граничные функции зависят от $ \varphi $.} $ u = u_1 +
u_2 $, где справедливы уравнения  
\[
  \begin{cases}
    \Delta u_1 = 0, \quad 0 < r < a, \ 0\leqslant \varphi \leqslant 2\pi,
    \ 0 < z < l, \\
    \alpha u_z(r, \varphi, l) + \beta u(r,\varphi, l) = f(r, \varphi),\\
    -\alpha_1 u_z(r,\varphi, 0) + \beta_1 u(r,\varphi, 0) = g(r, \varphi),\\
    \alpha_2 u_r(a,\varphi, z) + \beta_2 u(a,\varphi, z) = 0;
  \end{cases} 
  \begin{cases}
    \Delta u_2 = 0, \quad 0 < r < a, \ 0\leqslant \varphi \leqslant 2\pi,
    \ 0 < z < l, \\
    \alpha u_z(r, \varphi, l) + \beta u(r,\varphi, l) = 0,\\
    -\alpha_1 u_z(r,\varphi, 0) + \beta_1 u(r,\varphi, 0) = 0,\\
    \alpha_2 u_r(a,\varphi, z) + \beta_2 u(a,\varphi, z) = h(\varphi, z).
  \end{cases}
\]
Здесь уже нужно разделить переменные. Для обеих задач получаем
соотношения
%TODO: откуда знаем, что \lambda >= 0?
\[
  \frac{\Delta_{r\varphi} v(r, \varphi)}{v(r,\varphi)} = - \frac{Z''(z)}{Z(z)}
  =: \lambda.
\]
При этом для $ u_1 $ имеем $ \lambda \leqslant 0 $ ввиду однородности условий на $ v $,
а для $ u_2 $ имеем $ \lambda \geqslant 0 $ ввиду однородности\footnote{Задача
Штурма -- Лиувилля с однородными условиями нашего вида не может иметь
отрицательных собственных значений.} условий на $ Z $. Как и выше, получили по две задачи Штурма -- Лиувилля на каждое $ u $ --- на
отрезке и внутри круга. Более того, теперь у этих двух задач общие собственные
значения $ \lambda_k $, что позволяет найти их лишь однажды.
%\footnote{Сделать это
 % лучше в одномерной задаче, поскольку в круговой собственные значения находятся
%численно.}. 
Перечислим эти четыре задачи Штурма -- Лиувилля: 
\begin{align*}
  % &\begin{cases}
    &Z_1'' + \lambda Z_1 = 0, \quad 0 < z < l,
    % \alpha Z_1'(l) + \beta Z_1(l) = \ldots,\\
    % -\alpha_1Z_1'(0) -\beta_1 Z_1(0) = \ldots,
  % \end{cases} 
  &\begin{cases}
    \Delta v_1 - \lambda v_1 = 0, \quad 0 < r < a, \ 0 \leqslant \varphi \leqslant
    2\pi,\\
    \alpha_2 v_{1r}(a,\varphi) + \beta_2 v_1(a,\varphi) = 0;
  \end{cases} \\
  &\begin{cases}
    Z_2'' + \lambda Z_2 = 0, \quad 0 < z < l,\\
    \alpha Z_2'(l) + \beta Z_2(l) = 0,\\
    -\alpha_1Z_2'(0) -\beta_1 Z_2(0) = 0,
  \end{cases} \qquad
  % &\begin{cases}
    &\Delta v_2 - \lambda v_2 = 0, \quad 0 < r < a, \ 0 \leqslant \varphi \leqslant
    2\pi.
    % \alpha_2 v_{2r}(a,\varphi) + \beta_2 v_2(a,\varphi) = \ldots
  % \end{cases}
\end{align*}
Решение задачи $ v_1 $ нам уже известно. С известными $ \lambda_{nk} $ уравнение на $ Z $ будет иметь
гиперболические или линейные решения\footnote{См. раздел \ref{sec:S-L}.}.
Полученный ряд $ u_1 = \sum u_1^{(nk)} $, $ u_1^{(nk)} =
R_{nk}(r)\Phi_{n}(\varphi)Z_{k}(z)$
подставляем в оба граничных условия, попутно раскладывая функции $ f, \ g $ по
полученным левым частям. Приравнивая коэффициенты в левой и правой частях, получаем
ответ\footnote{Как, например, в решении из раздела
\ref{sec:laplace_example}.}.

% Как видно, ранее не рассматривались только случаи задач на $ v_1 $ и $ v_2 $. В обеих
% задачах изменился знак собственного значения, а у $ v_2 $ к тому же неоднородное
% граничное условие. На $ v_2 $ и остановимся.

Решение для $ Z_2 $ известно. Для $ v_2 $ после разделения переменных получим 
\[
    \frac{r(rR')' - \lambda r^2 R}{R} = -\frac{\Phi''}{\Phi} =: \nu \geqslant 0,
  \]
то есть уравнение на $ \Phi $ осталось без изменений, а для $ R $ теперь
имеем
\[
    r^2R'' + rR' - (\lambda r^2 + n^2)R = 0.
\]
Сравнивая с системой \eqref{eq:bessel_eq}, видим, что теперь воспользоваться
функциями Бесселя не получится. Однако практически той же заменой $ x :=
i\sqrt\lambda r $ приходим к уравнению Бесселя 
\[
    x^2 y'' + x y' + (x^2 - n^2)y = 0.
\]
Его решениями будут функции Бесселя от чисто мнимого аргумента, именуемые
\emph{функцией Инфельда $ I_n $} и \emph{функцией Макдональда $ K_n $}.
Последнюю исключаем, поскольку она неограниченна в нуле. Случай $ \lambda = 0 $
приводит к классическому уравнению Эйлера\footnote{Подробнее разобрано в других
разделах.}, которое подстановкой $ r^\varkappa $ даёт решения $ R(r) = A + B\ln
r$ при $ n = 0 $ и $ Ar^{n} + Br^{-n} $ при $ n \in \mathbb N $. Константа будет
учтена другими решениями, а решения $ r^{-n}, \ln r $ следует исключить ввиду их
неограниченности в нуле. Итоговая ФСР: $ \{r^n, I_n(\sqrt{\lambda_k} r)\} $.
Осталось представить решение в виде $ u_2^{(nk)} = R_{nk}(r) \Phi_n(\varphi) Z_k(z)
$, $ u_2 = \sum u_2^{(nk)} $ и разложить\footnote{Длину $ \|I_n\| $ можно найти тем же способом, каким
искалась длина $ \|J_n\| $.}  граничную функцию $ h(\varphi, z)
$ по базису $ \alpha_2u_r^{(nk)} + \beta_2 u $. Покоординатное приравние
коэффициентов даст ответ.
\subsubsection{Пуассон}
%ref: Пикулин, стр. 37
Для решения задачи Пуассона нужно сначала подобрать\footnote{Это возможно
  сделать при некотором специальном виде правой части. Например, для неоднородной части $ r^2\cos 2\varphi $ решение подбирается в виде
$ Ar^\alpha\cos2\varphi $; если неоднородная часть константа, то $ w = Ar^\alpha
$.} частное решение
неоднородного уравнения
(без учёта граничных условий) $ w(r,\varphi, z) $, а после перейти к однородной
задаче
\[
  \begin{cases}
    \Delta u = 0, & 0 < r < a, \ 0\leqslant \varphi \leqslant 2\pi,
    \ 0 < z < l,\\
    \alpha (u_z - w_z) + \beta (u-w) \Big|_{z=l} = f(r, \varphi),\\
      -\alpha_1 (u_z - w_z) + \beta_1 (u - w)\Big|_{z=0} = g(r, \varphi),\\
      \alpha_2 (u_r - w_r) + \beta_2 (u-w) \Big|_{r=a} = h(\varphi, z).
  \end{cases}
\]
