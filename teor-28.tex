\subsection{Фундаментальное решение дифференциального оператора. Обобщённое решение задачи Коши.}

Написанное ниже взято из Владимирова, страницы 144--147.

\paragraph{Фундаментальное решение}
ОПРЕДЕЛЕНИЕ. Пусть $L$ -- дифференциальный оператор с постоянными коэффициентами,
$a_\alpha (x) = a_\alpha = \operatorname{const}$,
\[
  L(\partial) = \sum_{|\alpha| = 0}^m a_\alpha \partial^\alpha, L^* (\partial) = L(-\partial).
\]
\emph{Фундаментальным решением (функцией влияния)} оператора $L(\partial)$ называется обобщенная
функция $\mathcal{E} \in \mathcal{D}' (\mathbb{R}^n)$, удовлетворяющая в $\mathbb{R}^n$ 
уравнению
\[
  L(\partial) \mathcal{E} = \delta (x).
\]

Оно не единственно и определяется с точностью до слагаемого
$\mathcal{E}_0: L(\partial) \mathcal{E}_0 = 0$.

Верна следующая лемма:
\begin{theorem}
  \[
    \mathcal{E} \in \mathcal{S}' \text{-- фундаментальное решение $L(\partial)$}
    \Leftrightarrow
    L(-i \xi) F[\mathcal{E}] = 1,
  \]
  где $L(\xi) = \sum_{|\alpha|=0}^m a_\alpha \xi^\alpha$, $F[\mathcal{E}]$ -- преобразование Фурье.
\end{theorem}

Эта лемма следует из свойств преобразования Фурье. Из этой леммы мы знаем о том, каким образом можно
находить фундаментальные решения, а так же знаем, что они не единственны в силу не единственности 
решений получаемых алгебраических уравнений. Так же было доказано, что получаемое уравнение всегда разрешимо в классе $\mathcal{S}'$, если $L(-i\xi)$ не равно тождественно нулю.


\begin{theorem}[Теорема о решении уравнения с правой частью]
  Пусть $f \in \mathcal{D}'$ такова, что $\exists \mathcal{E} * f \in \mathcal{D}'$. Тогда решение
  уравнения $L(\partial) u = f(x)$ существует в $\mathcal{D}'$ и даётся формулой
  \[
    u = \mathcal{E} * f.
  \]
  Причём это решение единственно в классе тех обобщённых функций из $\mathcal{D}'$, для которых
  существует свёртка с $\mathcal{E}$.
\end{theorem}
Эта теорема следует из свойств дифференцирования свёртки.



\paragraph{Обобщённое решение задачи Коши} \footnote{(Владимиров, Жаринов, стр 168)}
Сформулируем задачу Коши: оператор $L = \sum_{k=0}^m a_k \dfrac{d}{dx^k}$:
\[
  \begin{cases}
    Ly = f(x), x > 0, f \in \mathcal{C} \\
    y(0) = y_0, \\
    y'(0) = y_1, \\
    \dots \\
    y^{(m-1)} (0) = y_{(m-1)}.
  \end{cases}
\]
% TODO недописано
Продлим функции $y(x)$ и $f(x)$ на область $t < 0$ нулём. Продолжение этих функций обозначим
$\tilde y$, $\tilde f$. Тогда у функции $\tilde y$ в точке 0 имеет некий скачок, её производная
аналогично тоже и тд, причём величины этих скачков можно достать из начальных условий задачи Коши,
поэтому $\tilde y$ и все её производные представляются в виде
\footnote{непрерывная везде кроме 0 функция f со скачком в т. 0 величины $[f]_0 = f(+0) - f(-0)$ выражается как $f={f} + \theta(t) \cdot [f]_0$, где функция ${f}$ -- уже непрерывная везде}:
\[
  \tilde y^{(k)} = \left\{ \tilde y^{(k)} \right\} + \sum_{j=0}^{k-1} a_j \delta^{(k-j-1)} (t),
\]
тогда:
\[
  L\tilde y = {L\tilde y (t)} + y_0 \delta^{n-1}(t) + (a_1 y_0 + y_1) \delta^{n-2} (t) + \dots
  + (a_{n-1} y_0 + \dots + a_1 y_{n-2} + y_{n-1}) \delta(t) = \tilde f(t)
  + \sum_{k=0}^{n-1} c_k \delta^{(k)} (t),
\]
где
\[
  c_0 = a_{n-1} y_0 + \dots + a_1 y_{n-2} + y_{n-1}, \dots, c_{n-2} = a_1 y_0 + y_1, c_{n-1} = y_0.
\]

Таким образом, фунция $\tilde u$ в обобщённом смысле удовлетворяет в $\mathbb{R}^1$ дифференциальному
уравнению 
\[
  L\tilde y = \tilde f(t) + \sum_{k=0}^{n-1} c_k \delta^{(k)} (t).
\]

Обозначим за $Z(x)$ решение однородного уравнения:
\[
  \begin{cases}
    LZ = 0, \\
    Z(0) = Z'(0) = \dots = Z^{(m-2)} (0) = 0, \\
    Z^{(m-1)} (0) = 1.
  \end{cases}
\]
% (на всякий случай, такое решение будет в виде: $Z(x) = \sum C_k e^{\lambda_k x}$)

Очевидно, что $\mathcal{E} = \theta(x) Z(x)$ является фундаментальным решением оператора $L$, ибо
\[
  \mathcal{E}' = \theta' Z + \theta Z' = \theta Z', 
  \dots
  \mathcal{E}^{(m-1)} = \theta(t) Z^{(m-1)}(t),
  \mathcal{E}^{(m)} = \delta(t) + \theta(t) Z^{(m)} (t).
\]
ну и подставляя эти выражения в $L\mathcal{E}$ там сократится всё, останется только $\delta$ от 
$\mathcal{E}^{(m)}$.

Тогда решение диффура выражается как свертка с правой частью:
\[
  \tilde y(x) = \mathcal{E} * \left(\tilde f + \sum_{k=0}^{n-1} c_k \delta^{(k)}\right)
  = \mathcal{E} * \tilde f + \sum_{k=0}^{n-1} c_k \mathcal{E}^{(k)} (t)
  = \theta(t) \int\limits_0^{t} Z(t-\tau) f(\tau) \, d\tau
  + \theta(t) \sum_{k=0}^{n-1} c_k Z^{(k)} (t),
\]
и здесь мы использовали вот это:
\[
  \mathcal{E}^{(k)} = \left( \theta(t) Z(t) \right)^{(k)}
  = \theta(t) Z^{(k)} (t), k = 0, 1, \dots, n-1. 
\]

Теперь осталось только вернуться назад к $y(x)$, а для этого можно просто убрать $\theta(t)$.


\paragraph{Фундаментальные решения основных операторов}
\begin{center}
  \begin{tabular}{|c|c|}
    \hline
    Оператор & фундаментальное решение \\
    \hline
    теплопроводности $\dfrac{\partial \mathcal{E}}{\partial t} - a^2 \Delta \mathcal{E}$ &
    $\mathcal{E} (x, t) = \dfrac{\theta(t)}{(2a \sqrt{\pi t})^n} \exp \left\{ -\dfrac{|x|^2}{4a^2 t} \right\}$ \\
    
    \hline
    Лаплас $\Delta \mathcal{E}$ &
    $\mathcal{E}_2 = \dfrac{1}{2\pi} \ln |x|, \mathcal{E} = -\dfrac{1}{(n-2) \sigma_n} |x|^{-n+2}$ \\
    \hline 
    Гельмгольца $(\Delta + k^2) \mathcal{E}_n$ &
    $\mathcal{E}_1 (x) = \dfrac{1}{2ik} e^{ik |x|}, \bar{\mathcal{E}}_1 (x) = - \dfrac{1}{2ik} e^{-ik|x|}$ \\
    \hline
  \end{tabular}
\end{center}


