\subsection{Пространство быстроубывающих функций и пространство функций медленного роста. Обобщённое преобразование Фурье. Обобщённое преобразование Фурье свертки и обобщённое равенство Парсеваля.}

\paragraph{Пространства $\mathcal{D}$ и $\mathcal{D}'$}
ОПРЕДЕЛЕНИЕ. Отнесём ко множеству \emph{основных функций} $\mathcal{D} = \mathcal{D} (\mathbb{R}^n)$ все
финитные бесконечно дифференцируемые в $\mathbb{R}^n$ функции. Сходимость в $\mathcal{D}$
определим следующим образом:
\[
  \varphi_k \to \varphi \text{ в } \mathcal{D}, \varphi_k, \varphi \in \mathcal{D}
  \Leftrightarrow
  \begin{cases}
    \exists R > 0 : \operatorname{supp} \varphi_k \subset U_R, \\
    \forall \alpha=(\alpha_1, \alpha_2, \dots, \alpha_n) :
      \partial^{\alpha} \varphi_k(x) \rightrightarrows \partial^{\alpha} \varphi(x), x\in\mathbb{R}^n, 
      k \to \infty
  \end{cases}
\]

Обозначение: $\mathcal{D} (G) = \left\{ \varphi \in \mathcal{D} : \supp\varphi \subset G \right\}$.

ОПРЕДЕЛЕНИЕ. Обобщённой функцией называется всякий линейный непрерывный функционал на пространстве
основных функций $\mathcal{D}$. $\mathcal{D}' = \left\{ f: \mathcal{D} \mapsto \mathbb{C} \right\} $

\begin{itemize}
  \item Линейный: $\forall f \in \mathcal{D}' \quad \forall \varphi, \psi \in \mathcal{D} : (f, \lambda \varphi + \mu \psi) = \lambda (f, \varphi) + \mu (f, \psi)$.

  \item Непрерывный: $\varphi_k \to 0 \text{ в } \mathcal{D}, k \to \infty \Rightarrow (f, \varphi_k) \to 0, k \to \infty$.

  \item Сходимость в пространстве $\mathcal{D}'$: $f_k \to f, f_k, f \in \mathcal{D}' \Leftrightarrow
\forall \varphi \in \mathcal{D} : (f_k, \varphi) \to (f, \varphi), k \to \infty$.

 \item Говорят, что обобщённая функция равна нулю в $G$, если $\forall \varphi \in \mathcal{D}(G) : (f, \varphi) = 0$. 
\end{itemize}

ОПРЕДЕЛЕНИЕ. \emph{Регулярными обобщёнными функциями} называются такие обобщённые функции, которые
определяются локально интегрируемыми в $\mathbb{R}^n$ функциями по формуле:
\[
  (f, \varphi) = \int f(x) g(x) \, dx,
\]
остальные называются \emph{сингулярными}.

То есть таким функциям можно сопоставить какуюто нормальную обычную функцию $f(x)$, причём верна
лемма Дюбуа-Реймона: для того, чтобы локально интегрируемая в $G$ функция $f(x)$ обращалась в нуль
в области $G$ в смысле обобщенных функций, необходимо и достаточно, чтобы $f(x) = 0$ в $G$.


